\documentclass{ctexbook}%ctexart

%%\usepackage{ctex}
%\ctexset{	%	设置格式
%	section = {
%		format += \zihao{-4} \heiti \raggedright,
%		name = {,、},
%		number  = \chinese{section},
%		beforeskip = 1.0ex plus 0.2ex minus .2ex,
%		afterskip = 1.0ex plus 0.2ex minus .2ex,
%		aftername = \hspace{0pt}
%	},
%	subsection ={
%		format += \zihao{-4} \heiti \raggedright,
%		name = {,、},
%		number  = \chinese{section},
%		beforeskip = 1.0ex plus 0.2ex minus .2ex,
%		afterskip = 1.0ex plus 0.2ex minus .2ex,
%		aftername = \hspace{0pt}
%	}
%}


%	正文
\begin{document}
	\tableofcontents
	
	\chapter{绪论}	%	章节大纲,需要使用ctexbook,subsub不起作用
	%	小节
	\section{引言}	% 小节
	%	\paragraph{} 表示行
	超声波是一种频率高于20000赫兹的声波,它的方向性好,反射能力强,易于获得较集中的声能,在水中传播距离比空气中远,可用于测距、测速、清洗、焊接、碎石、杀菌消毒等。在医学、军事、工业、农业上有很多的应用。
	
	超声波因其频率下限超过人的听觉上限而得名。科学家们将每秒钟振动的次数称为声音的频率,它的单位是赫兹(Hz)。
	
	我们人类耳朵能听到的声波频率为20Hz-20000Hz。因此,我们把频率高于20000赫兹的声波称为“超声波”。通常用于医学诊断的超声波频率为1兆赫兹-30兆赫兹。
	
	\chapter{研究方法}
	\section{实验目的}
	\subsection{目的一:}	%	子小节
	\subsection{目的二:}
	\subsubsection{意义:}	%	子子小节 
	\section{实验方法}
	
	\chapter{总结}
	\section{优点}
	\section{不足}
\end{document}