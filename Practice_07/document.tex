\documentclass{ctexart}

\usepackage{amsmath}

\begin{document}
	\section{简介}
		\LaTeX{}文档分为数学模式和文本模式
		
	\section{行内公式}
	\subsection{美元符号}
		$a + b = c$
	\subsection{小括号}
		\( a + b = c \)
	\subsection{math环境}
		\begin{math}
			 1 + 2 = 3
		\end{math}
	\section{上下标}
	\subsection{上标}
		$X^2 X^{29}$
	\subsection{下标}
		$X_2 X_{34}$
	\section{希腊字母}
		$\alpha	\beta	\gamma	\pi	\omega$	
		
		$\Gamma	\Delta	\Theta	\Pi	\Omega$ 
	\section{数字函数}
		$\log$		$\sin$		$\cos$		$\arccos$	$\arcsin$		$\arctan$		$\ln$
		
		$y = log_{10} 10$
		
		$y = \sqrt{2}$
		
	\section{分式}
		$3/4$	%	横
		$\frac{3}{4}$	%	竖
		
	\section{段间公式}
	\subsection{美元符号}
		$$a+b=c$$
	\subsection{中括号}
		\[a + b = c\]
	\subsection{displaymath环境}
		\begin{displaymath}
			1 + 2 = 3
		\end{displaymath}
	\subsection{动编号的equation}
		见式\ref{eq:formulate_01}:
		\begin{equation}
			1 + 2 = 3
			\label{eq:formulate_01}
		\end{equation}
	\subsection{无编号的equation}	%	需要宏包
		见式\ref{eq:formulate_02}:	
		\begin{equation*}
			1 + 2 = 3
			\label{eq:formulate_02}
		\end{equation*}
	
\end{document}